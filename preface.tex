\section*{Preface}
\addcontentsline{toc}{section}{\protect\numberline{}Preface}%

These are the course notes to accompany \emph{CMSC631 Program Analysis
  and Understanding} for the Spring semester of 2014 at the University
of Maryland.
%
This course has been taught many times, over several years, by a
variety of distinguished and seasoned professors, who have put
together a wide array of fine supplementary materials, all of which
have been foolishly discarded by your current professor, who is
neither distinguished, nor seasoned.  ``Why?'' is the perfectly
reasonable thought that should be running through your mind at this
point.

Bret Victor, one of the more interesting philosophers of programming
around today, wrote a little
essay\footnote{\url{http://worrydream.com/SomeThoughtsOnTeaching/}} on
teaching in which he said:
%
\begin{quote}
When I write or talk, it comes out of trying to understand a way of
thinking that's deeply personal and valuable to me, and then trying to
share this understanding. It's more than mere passion --- anyone can
be passionate about anything. It's a kind of honesty that comes from
distilling and passing on \emph{my own genuine insights and
  experience}.
\end{quote}
And so that's really the reason your unseasoned, undistinguished
professor has thrown it all out.  These notes are an approximation of
his own insights and experiences, and are surely full of mistakes and
shortcomings, reflecting gaps in his own understanding.

Since your professor has thrown out all of the prepared materials
\emph{and} this is the first time he's taught this course, these notes
will unfortunately be prepared in a \emph{just-in-time} fashion.  As a
kind of apology, the source code for these notes have been made
available in a public repository.  If you spot errors, have better
ways of presenting ideas, or would like to raise questions not
answered in the notes, you are encouraged to create issues and pull
requests:

\begin{center}
\url{https://github.com/cmsc631/notes}
\end{center}

Contributions to the notes will be acknowledged explicitly in the
text, and implicitly in the participation component of your grade.





