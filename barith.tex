\newcommand\err{\mathbf{err}}
\newcommand\prop{\mathbf{prop}}

\newcommand\compat[1]{\rightarrow_{#1}}
\newcommand\multicompat[1]{\rightarrow^\star_{#1}}
\newcommand\std[1]{\longmapsto_{#1}}
\newcommand\multistd[1]{\longmapsto^\star_{#1}}

\newcommand\bep{\mathbf{bep}}
\newcommand\be{\mathbf{be}}

\newcommand\envreduce[4][\menv]{#1 \vdash {#2\;#3\;#4}}

\newcommand\unaryerror[2]{
\inferrule{\beval\menv{\mexp}\mval \\ \mval \not\in {#2}}
          {\beval\menv{{#1}(\mexp)}{\Err_{#1}}}
}

\newcommand\binaryerror[2]{
\inferrule{\beval\menv{\mexp_1}\mval \\ \mval \not\in {#2}}
          {\beval\menv{{#1}(\mexp_1,\mexp_2)}{\Err_{{#1}1}}}

\inferrule{\beval\menv{\mexp_2}\mval \\ \mval \not\in {#2}}
          {\beval\menv{{#1}(\mexp_1,\mexp_2)}{\Err_{{#1}2}}}
}

\newcommand\binaryreduceerror[3]{
\inferrule{\mval \not\in {#3} }
          {\envreduce{{#2}(\mexp,\mval)}{#1}{\Err_{{#2}1}}}

\inferrule{\mval \not\in {#3} }
          {\envreduce{{#2}(\mval,\mexp)}{#1}{\Err_{{#2}2}}}
}

\newcommand\binaryreducevalerror[3]{
\inferrule{\mval_1 \not\in {#3} }
          {\envreduce{{#2}(\mval_1,\mval_2)}{#1}{\Err_{{#2}1}}}

\inferrule{\mval \not\in {#3} }
          {\envreduce{{#2}(\mint,\mval)}{#1}{\Err_{{#2}2}}}
}


\section{Variables, Conditionals, \& Errors}

The $\Arith$ language is about as simple a programming language as
you're likely to find, although it served us well in demonstrating many
of the core ideas behind syntax and semantics.
%
As programmers, we like programming in rich, expressive languages, 
which $\Arith$ ain't.
%
But as we enrich languages, we must also enrich our models.
%
Let's look at a minimal increment to the $\Arith$ language and what it
entails for our models.
%
Let's add a new binary operator, $\Div$, which performs integer
division.
%
Let's add a new binary operator, $\Eq$, which determines if two
integers are equal.
%
Since an answer to the question ``are these two numbers equal?'' is
either a yes or a no, let's add a new kind of value, Booleans, to the
language to represent such answers.
%
Booleans are useful for conditionally selecting computations, let's
add a conditional form, $\If(e_1,e_2,e_3)$, which will select $e_2$
when $e_1$ evaluates to $\True$ and $e_3$ when it evaluates to
$\False$.  Finally, for good measure, let's throw in variables so that
programs run on given inputs.  We'll call this language $\Barith$ to
distinguish it from $\Arith$.

The syntax of $\Barith$ is straightfoward to define.  We add a couple
nullary constructors for Booleans, a binary constructor for equality
and for division, and a ternary constructor for conditionals.  For
variables, we assume there is an infinite, enumerable set of variable
names, $\Var$, and use lower-case, italic names to denote elements of
that set.  Two variables are considered ``the same'' if they are
spelled the same:

\[
\begin{array}{llcl}
\mathit{Var} & \mvar & = & \s{x}\ |\ \s{y}\ |\ \s{z}\ |\ \dots \\
\mathit{Bool} & \mbool & = & \True\ |\ \False\\
\Barith & e & = & \mvar\ |\ \mbool\ |\ \Eq(e,e)\ |\ \Div(e,e)\ |\ \If(e,e,e)\\
        &   & | & \mint\ |\ \Pred(e)\ |\ \Succ(e)\ |\ \Plus(e,e)\ |\ \Mult(e,e)
\end{array}
\]



Before delving into the details of the semantics, it's worth taking
time to ponder over a few $\Barith$ programs and think informally
about what they should mean:

\begin{enumerate}\setlength{\itemsep}{0pt}
\item $\s{x}$
\item $\Eq(\False,7)$
\item $\Div(7,0)$
\item $\If(4,1,2)$
\item $\If(\False,\Div(7,0),3)$
\end{enumerate}

Let's take the first one.  What should $\s{x}$ mean?  Consider this
question in the context of a mathematical formula: $x^2+4$.  What does
$x$ mean here?  The answer is: it depends.  You can answer the
question without first giving a meaning to the variable.  So $x^2+4$
means $13$ when $x=3$ and it means $29$ when $x=5$.  The meaning of
the formula varies with the meaning of the variable.  The same should
be true for our programming language.

What about $\Eq(\False,7)$?  Here you could make different design
choices.  You might follow the JavaScript approach and produce false
since $\False$ is different than $7$.  Or maybe true because the
pointer representation of $7$ and $\False$ turn out to be the same.
Or maybe you return $42$, because why not?  Or you might do something
sensible and raise an error because a numerical operator was applied
to the wrong kind of argument?

In this section, we are going to take the simplest path forward.  We
will only consider well-behaved programs.  In the case of the natural
semantics, this means programs such as $\If(4,1,2)$ are literally
\emph{meaningless}: the semantic relation will be undefined on such
programs.  In the case of the reduction semantics, such programs ``get
stuck'': they reach a bad state that cannot proceed further; errors
manifest as irreducible programs that are not values.

\subsection{Natural Semantics}

Let's identify a set of \emph{values} and a set of \emph{answers}.
Values include integers and Booleans.  Answers, which represent the
final results of evaluation are for the moment just values:
\[
\begin{array}{llcl}
\mathit{Val} & \mval & = & \mbool\ |\ \mint\\
\mathit{Ans} & \mans & = & \mval\\
\end{array}
\]


The evaluation of programs containing variables depends on an
environment which gives meaning to these variables.  In the natural
semantics is modelled as a ternary relation $(\cdot \vdash \cdot
\Downarrow \cdot) \subseteq (\mathit{Var} \rightharpoonup
\mathit{Val})\times \Barith \times \mathit{Ans}$.  The relation
includes an \deftech{environment} which maps variables to values.
Environments, ranged over by meta-variable $\menv$, are partial
functions from variables to values.  Values are self
evaluating. Variables evaluate to the value given by the environment.
Division performs the integer division of its arguments when they
evaluate to numeric values and the denominator is non-zero. Equality
is defined as you might expect, producing true when its arguments
evaluate the same number, false when they evaluate to different
numbers.  Conditionals select which value to produce based on the
evaluation of the test expression.
\begin{mathpar}
\inferrule{\ }
          {\beval\menv\mval\mval}

\inferrule{\menv(\mvar) = \mval}
          {\beval\menv\mvar\mval}

\inferrule{\beval\menv{\mexp_1}\mint
  \\ \beval\menv{\mexp_2}\moint
  \\ \moint \neq 0
  \\ k = \lfloor \mint / \moint \rfloor}
          {\beval\menv{\Div(\mexp_1,\mexp_2)}{k}}

\inferrule{\beval\menv{\mexp_1}\mint \\ \beval\menv{\mexp_2}\moint \\ \mint = \moint}
          {\beval\menv{\Eq(\mexp_1,\mexp_2)}\True}

\inferrule{\beval\menv{\mexp_1}\mint \\ \beval\menv{\mexp_2}\moint \\ \mint \neq \moint}
          {\beval\menv{\Eq(\mexp_1,\mexp_2)}\False}

\inferrule{\beval\menv{\mexp_1}\True \\
  \beval\menv{\mexp_2}\mans}
          {\beval\menv{\If(\mexp_1,\mexp_2,\mexp_3)}\mans}

\inferrule{\beval\menv{\mexp_1}\False \\
  \beval\menv{\mexp_3}\mans}
          {\beval\menv{\If(\mexp_1,\mexp_2,\mexp_3)}\mans}
\end{mathpar}
We can also assume the omitted rules for handling expressions from
$\Arith$, which are just like before but carrying an environment.

The important to note here is what is missing.  There's no definition
for what happens, for example, when $\mvar$ is not defined in $\menv$,
nor when $\moint=0$ in the rule for $\Div$.  Likewise, a program that
compares non-numeric values for equality will not have an answer
according to $\Downarrow$.

This semantics still has the nice property of giving the same answer
whenever it gives an answer, but unlike the natural semantics for
$\Arith$, it does not give an answer for all programs.  It is
therefore a partial function.

\subsection{Compiling, without Errors}

Here's a compiler for the $\Barith$ language, minus errors.

\begin{verbatim}
type barith =
  | Int of int
  | Pred of barith
  | Succ of barith
  | Plus of barith * barith
  | Mult of barith * barith
  (* new stuff *)
  | Var of string
  | Bool of bool
  | If of barith * barith * barith
  | Div of barith * barith

type value =
  | VInt of int 
  | VBool of bool
  | VError of string

type env = (string * value) list

let rec lookup (r : env) (x : string) : value =
  match r with
    | (y,v)::r' -> if y=x then v else lookup r' x

let rec compile (e : barith) : (env -> value) =
  match e with
    | Int i -> (fun _ -> VInt i)
    | Bool b -> (fun _ -> VBool b)
    | Pred e ->
        let c = compile e in
        (fun env ->
           let VInt i = c env in
             VInt (i-1))
    | Succ e ->
        let c = compile e in
          (fun env -> 
             let VInt i = c env in
               VInt (i+1))
    | Mult (e1, e2) ->
        let c1 = compile e1 in
        let c2 = compile e2 in
          (fun env ->
             let VInt i = (c1 env) in
             let VInt j = (c2 env) in
               VInt (i*j))
    | Plus (e1, e2) ->
        let c1 = compile e1 in
        let c2 = compile e2 in
          (fun env ->
             let VInt i = (c1 env) in
             let VInt j = (c2 env) in
               VInt (i+j))
    | Div (e1, e2) ->
        let c1 = compile e1 in
        let c2 = compile e2 in
          (fun env ->
             let VInt i = (c1 env) in
             let VInt j = (c2 env) in
               VInt (i/j))
    | If (e1, e2, e3) ->
        let c1 = compile e1 in
        let c2 = compile e2 in
        let c3 = compile e3 in
          (fun env ->
             let VBool b = c1 env in
               if b=true then c2 env else c3 env)
    | Var x ->
        (fun env -> lookup env x)
\end{verbatim}


\subsection{Reduction Semantics}

Modelling $\Barith$ with reduction semantics is pretty easy.  The
reduction axioms are just the following (and we assume $\reduce$-like
axioms, too):
\begin{mathpar}
  \inferrule{\menv(\mvar) = \mval}
            {\breduce\menv\mvar\mval}

  \inferrule{\mint=\moint}
            {\breduce\menv{\Eq(\mint,\moint)}\True}

  \inferrule{\ }
            {\breduce\menv{\If(\True,\mexp_1,\mexp_2)}{\mexp_1}}

  \inferrule{\ }
            {\breduce\menv{\If(\False,\mexp_1,\mexp_2)}{\mexp_2}}

  \inferrule{\mint\neq \moint}
            {\breduce\menv{\Eq(\mint,\moint)}\False}

  \inferrule{j\neq 0 \\ k = \lfloor \mint/\moint \rfloor}
            {\breduce\menv{\Div(\mint,\moint)}{k}}
\end{mathpar}

Notice that like the natural semantics, the meaning of a program is
given in terms of an environment.

The $\bstepname$ relation is the closure of $\breducename$ for
compatibility with the syntax of expressions; The $\bmultistepname$
relation is the relfexive, transitive closure of the $\bstepname$
relation.  The $=_\breducename$ equivalence relation is the symmetric
closure of the $\bmultistepname$ relation.  The syntactic theory is
consistent and standard reduction relation $\bstdstepname$ can be
defined as the contextual closure of $\breducename$ over the following
grammar of evaluation contexts (plus the grammar of $\mectx$ for
$\Arith$):
\[
\begin{array}{llcl}
\mathit{EvalContext} & \mectx & = & \dots \\
                 &       & | & \Eq(\mectx,\mexp)\ |\ \Eq(\mval,\mectx)\\
                 &       & | & \Div(\mectx,\mexp)\ |\ \Div(\mval,\mectx)\\
                 &       & | & \If(\mectx,\mexp,\mexp)
\end{array}
\]
The usual standardization property holds for $\Barith$.

Unlike the natural semantics, we can give a partial meaning to
programs that reach bad states because they reduce until the reduction
axiom cannot be applied, for example:
\begin{mathpar}
\bclosedstep{\Plus(3,\Mult(2,\Div(5,\Pred(1))))}
            {\Plus(3,\Mult(2,\Div(5,0)))}
\end{mathpar}
The expression on the right cannot take any further steps.  Since it
is not a value and cannot step further, we say it is \deftech{stuck}.



%%%%% 
\subsection{Meaningful Errors}

The development of the previous section, which essentially ignores
errors in the semantic models of $\Barith$, can be a useful and simple
way of conceiving of a programming language.  It can also lead to
issues, which depending on the setting, can be problematic.  Undefined
behavior in languages such as C are the source of major security
vulnerabilities, for example.  Sometimes it is worth modelling errors
explicitly.  For example, we may want to know what kinds of errors are
possible, and if errors are possible, which code component is at fault
for the error.  In the remainder of this section, we look at various
ways of modelling errors for our simple language.  Although this is a
simpified setting, we will see that errors, which are a primitive form
of \deftech{effect}, complicate reasoning about programs.  In
particular, we must be careful in designing the syntactic theory in
order to remaing consistent.


For a proper account of error behavior, we need to designate a set of
errors and include them in the set of answers that a program may compute:
\[
\begin{array}{llcl}
\mathit{Ans} & \mans & = & \dots\ |\ \merr\\
\mathit{Err} & \merr & = & \Err_\ell \text{ where $\ell$ is in some set}
\end{array}
\]
We'll assume there are an infinite number of different errors and
write subscripts to distinguish them ($\ell$ ranges over these subscripts).



\subsection{Natural Semantics for Imprecise Errors}

Errors can arise when a variable is unbound (it is not defined in the
environment), when division is performed with a denominator of 0, or
when the wrong kind of value is used for an operation, such as adding
a Boolean and an integer or using $\If$ with a non-Boolean test.
%
A variable is unbound when it is not in the domain of the environment;
formally: $\dom(\menv) = \{ \mvar\ |\ \menv(\mvar) = \mval \text{ for
  some }\mval \}$.
%
The following rules define when an error is created:
\begin{mathpar}
\inferrule{\mvar \notin \dom(\menv)}
          {\beval\menv\mvar{\Err_{\mvar}}}

\binaryerror\Eq\Int

\inferrule{\beval\menv{\mexp_1}\mint
  \\ \beval\menv{\mexp_2}\moint
  \\ \moint = 0}
          {\beval\menv{\Div(\mexp_1,\mexp_2)}{\Err_{\Div0}}}

\binaryerror\Div\Int

\inferrule{\beval\menv{\mexp_1}\mval \\ \mval \not\in \Bool}
          {\beval\menv{\If(\mexp_1,\mexp_2,\mexp_3)}{\Err_{\If}}}
\end{mathpar}
For completeness, we should also give the rules for expressions from
$\Arith$:
\begin{mathpar}
\unaryerror\Pred\Int

\unaryerror\Succ\Int

\binaryerror\Plus\Int

\binaryerror\Mult\Int
\end{mathpar}

While the above rules take care of creating errors when they arise, no
rules have dealt with the propagation of errors.  For example, we
should expect $\Eq(\Div(3,0),4)$ to produce a divide by zero error,
although as it stands the semantics is undefined.  Adding the
following rules propagates errors from subexpressions in $\Eq$:
\begin{mathpar}
\inferrule{\beval\menv{\mexp_1}\merr}
          {\beval\menv{\Eq(\mexp_1,\mexp_2)}\merr}

\inferrule{\beval\menv{\mexp_2}\merr}
          {\beval\menv{\Eq(\mexp_1,\mexp_2)}\merr}
\end{mathpar}
We need similar rules for all of the remaining syntactic constructors.
%% ,
%% with the exception of $\If$.  For $\If$, only the following rule is
%% added:
%% \begin{mathpar}
%% \inferrule{\beval\menv{\mexp_1}\Err}
%%           {\beval\menv{\If(\mexp_1,\mexp_2,\mexp_3)}\Err}>
%% \end{mathpar}
%% Why?  (Hint: think about the last example $\Barith$ program you were
%% asked to ponder.)

\paragraph{Properties of the semantics:}
Now that the semantics is in place, let's step back and consider some
of its properties.  The addition of errors has significantly
complicated the formal development---we had to add a bunch of rules
for signal and propogating errors.  But dealing with errors and
establishing the meaning of errors is a crucial aspect of semantic
engineering.  Remember that a semantics is useful for definining
properties of programs, and one of the more useful properties is
``this program causes no errors.''  So to even talk about this, we
need to confront error behaviors in the semantics.

But beyond the additional rules, errors, at least as we have
formulated them, have also caused an important change to our
semantics.  Unlike in the error-free case of $\Arith$, our evaluation
relation is no longer a function.  To see this, consider
$\Eq(\Div(3,0),\False)$.  This expression either produces a divide by
zero error ($\Err_{\Div0}$), or a type error ($\Err_{\Eq2}$) for giving a
Boolean to $\Eq$.  This is concerning for anyone who wants to write an
interpreter for, or reason about, $\Barith$ programs.  If we were
writing an interpreter, we should wonder ``what should it produce for
$\Eq(\Div(3,0),\False)$?''  One answer is an interpreter only has to
find \emph{some} answer that is consistent with the semantics; so if
$\syntax{eval}\; \menv\; \mexp = \mval$, then $\beval\menv\mexp\mval$,
but the reverse direction would not hold (the interpreter then, is an
abstraction of the natural semantics).  Many languages take this
approach.  Haskell for example, uses an imprecise exception
semantics\footnote{\url{http://research.microsoft.com/en-us/um/people/simonpj/papers/imprecise-exn.htm}},
basically following the outline we have above.  Other languages choose
to \emph{specify} which error should be signalled by determinizing the
language and restoring the functional nature of evaluation.

\subsection{Natural Semantics for Precise Errors}

The thing leading to the observed non-determinism of the evaluator is
the overlap in hypothesis in the different inference rules.  Consider the error
propagation rules for $\Eq$.  It's possible that
$\beval\menv{\mexp_1}\mbool$ \emph{and}
$\beval\menv{\mexp_2}{\mbool'}$, thus either rule could apply leading
to distinct answers.  By grade school analogy, it's as if the answer
to a basic algebra problem depended on the order in which you
simplified terms.  A solution then is to specify more rigorously the
order in which subexpressions should be evaluated, which can be
achieved by making the hypotheses of the inference rules
non-overlapping.  For example, if we want to evaluate $\Eq$
expressions left-to-right, we can use the following rules for error
generation and propogation:

\begin{mathpar}
\inferrule{\beval\menv{\mexp_1}\mbool \\
           \beval\menv{\mexp_2}\mval}
          {\beval\menv{\Eq(\mexp_1,\mexp_2)}{\Err_{\Eq1}}}

\inferrule{\beval\menv{\mexp_1}\mint \\
           \beval\menv{\mexp_2}\mbool }
          {\beval\menv{\Eq(\mexp_1,\mexp_2)}{\Err_{\Eq2}}}
\end{mathpar}

\begin{exercise}
Implement an interpreter for the imprecise error semantics as a
function in OCaml.  Your interpreter is free to produce any answer
that is consistent with the semantics.
\end{exercise}

\begin{exercise}\label{ex:determ}
Following the idea above, develop an alternative semantics for
$\Barith$ that is deterministic, i.e.  the evaluation relation is a
function.  Prove that it is in fact a function.  Implement this
function in OCaml.  Is this function a satisfying solution to the
previous problem?
\end{exercise}


\subsection{Consistent Syntactic Theory}

Now that we've seen the complications errors introduce in the setting
of natural semantics, let's look at the same development using
reduction semantics.

First, we will need to include $\Err$s in the syntax of $\Barith$
programs to model reduction:
\[
\begin{array}{llcl}
\mathit{Exp} & \mexp & = & \dots\ |\ \merr
\end{array}
\]


Next, let's define two reduction axioms.  The first, which we'll call
$\err$, creates errors.  The second, $\prop$, propagates them.
Some of rules for $\err$ are obvious:
\begin{mathpar}
  \inferrule{\mvar \notin \dom(\menv)}
            {\envreduce\mvar\err{\Err_\mvar}}

  \inferrule{\ }
            {\envreduce{\Div(\mint,0)}\err{\Err_{\Div0}}}

  \inferrule{\mval \not\in \Bool}
            {\envreduce{\If(\mval,\mexp_1,\mexp_2)}\err{\Err_\If}}

  \inferrule{\mval \not\in \Int}
            {\envreduce{\Pred(\mval)}\err{\Err_\Plus}}

  \inferrule{\mval \not\in \Int}
            {\envreduce{\Succ(\mval)}\err{\Err_\Succ}}

\end{mathpar}

But we would be wrong to add the following axioms:
\begin{mathpar}
  \binaryreduceerror\err\Plus\Int

  \binaryreduceerror\err\Mult\Int

  \binaryreduceerror\err\Eq\Int

  \binaryreduceerror\err\Div\Int
\end{mathpar}
The problem with these rules is they would immediately make the
syntactic theory inconsistent and identify all errors as equal.  For
example, we can have $\envreduce[\relax]{\Eq(\Err_1,\Err_2)}\err{\Err_1}$ and
$\envreduce[\relax]{\Eq(\Err_1,\Err_2)}\err{\Err_2}$, which induces the
equivalence $\vdash\Err_1 =_\err \Err_2$.  We also have to be
careful with how errors are propagated.  At first glance, you might
think to propagate errors past each of the syntactic constructors.  So
for example, you might have the following propagation rules for $\Eq$,
and similar rules for all the other constructors:
\begin{mathpar}
\inferrule{\ }
          {\envreduce{\Eq(\merr,\mexp)}\prop{\merr}}

\inferrule{\ }
          {\envreduce{\Eq(\mexp,\merr)}\prop{\merr}}
\end{mathpar}
But this also causes inconsistency.  Moreover, any program which
\emph{syntactically} contains an error expression (or an expression
that reduces to an error) would be equated with that error in the
theory.  That's not a useful equational theory.

Instead, we want to limit the $\err$ relation in the following
way.  In order for a type error to be created, the subexpressions must
be values:
\begin{mathpar}
  \binaryreducevalerror\err\Plus\Int

  \binaryreducevalerror\err\Mult\Int

  \binaryreducevalerror\err\Eq\Int

  \binaryreducevalerror\err\Div\Int
\end{mathpar}
Likewise, when propagating errors, we only allow propagation
when subexpressions to the left of the error are values:
\begin{mathpar}
\inferrule{\ }
          {\envreduce{\Eq(\merr,\mexp)}\prop{\merr}}

\inferrule{\ }
          {\envreduce{\Eq(\mval,\merr)}\prop{\merr}}
\end{mathpar}
The remaining rules for $\Pred$, $\Succ$, $\Div$, $\Plus$, and $\Mult$
are similar.  One notable exception is $\If$.  We only allow errors to
propagate past an $\If$ if the error occurs in the test position:
\begin{mathpar}
\inferrule{\ }
          {\envreduce{\If(\merr,\mexp_1,\mexp_2)}\prop{\merr}}
\end{mathpar}

We can now define a small-step reduction relation.  Let's call it
$\compat\bep$ to distinguish it from $\bstepname$.  We define
$\compat\bep$ as the compatible closure of $(\breducename\; \cup\; \err\;
\cup\; \prop)$ over the grammar of $\Barith$ expressions;
$\multicompat\bep$ is the reflexive transitive closure of
$\compat\bep$, and $=_\bep$ as the symmetric closure of
$\multicompat\bep$.

Note that this theory allows us to reduce inside of $\If$ expressions,
but if an error occurs, we cannot propogate it out until the test
expression produces a value and the $\If$ is reduced to the consequent
or alternative based on it.

\begin{exercise}[Consistency]
Prove if $\menv \vdash \mexp =_\bep \mans \mbox{ and } \menv
\vdash \mexp =_\bep \mans'$, then $\mans = \mans'$.
\end{exercise}

\begin{exercise}
Prove $\envreduce\mexp{\multicompat\bep}\mval \iff \envreduce\mexp{\multicompat\breducename}\mval$.
\end{exercise}


The approach of using evaluation contexts to specify canonical
reductions scales up to more complicated languages such as $\Barith$
with errors.  In fact, the notion of evaluation contexts does not
change from what has already been defined for $\Barith$, so the
standard reduction relation can be defined as:
\begin{mathpar}
\inferrule{\envreduce\mexp{(\breducename\; \cup\; \err\; \cup\; \prop)}{\mexp'}}
          {\envreduce{\plug\mectx\mexp}{\std\bep}{\plug\mectx{\mexp'}}}
\end{mathpar}


\begin{exercise}[Standardization]\label{ex:standardization}
Prove $\envreduce\mexp{\multistd\bep}\mans \iff \envreduce\mexp{\multicompat\bep}\mans$.
\end{exercise}

\begin{exercise}
Implement the standard reduction relation as a function in OCaml.
Implement an interpreter for $\Barith$ by iterating the standard
reduction relation until an answer is produced.
\end{exercise}


\subsection{Discarding the Context}

When introducing contexts in section~\ref{sec:context}, we noted that
formulating reductions explicitly in terms of contexts gives us
expressive power to write new kinds of reductions that are difficult
to do without a context.

To see an example, consider how programs produce errors
in our reduction semantics.  We have some expression, say
$\Plus(3,\Mult(2,\Div(5,0)))$, which introduces an error on the next
step and then step by step propagates the error outward until it is
the final answer:
\begin{mathpar}
\Plus(3,\Mult(2,\Div(5,0))) \;\compat\bep\;
\Plus(3,\Mult(2,\Err_{\Div0})) \;\compat\bep\;
\Plus(3,\Err_{\Div0}) \;\compat\bep\;
\Err_{\Div0}
\end{mathpar}
Achieving this step-by-step bubbling up of errors tedious because it
required a bunch of reduction axioms of the form
$\Mult(\mval,\Err_\ell)\;\prop\;\Err_\ell$.
%
We can avoid the propagating reductions and intermediate steps in the
standardized semantics.

Consider the following alternative standard reduction relation:
\begin{mathpar}
\inferrule{\envreduce\mexp{(\breducename\; \cup\; \err)}{\mexp'}}
          {\envreduce{\plug\mectx\mexp}{\std{\be}}{\plug\mectx{\mexp'}}}

\inferrule{\mectx \neq \hole }
          {\envreduce{\plug\mectx{\Err_\ell}}{\std{\be}}{\Err_\ell}}
\end{mathpar}
The example now immediately jumps to the final error state:
\begin{mathpar}
\envreduce{\Plus(3,\Mult(2,\Err_{\Div0}))}
          {\std{\be}}
          {\Err_{\Div0}}\text.
\end{mathpar}


\begin{exercise}
Prove $\envreduce\mexp{\multistd{\be}}\mans \iff
\envreduce\mexp{\multistd{\bep}}\mans$.
\end{exercise}

